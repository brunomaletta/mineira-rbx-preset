\documentclass[a4paper,11pt]{article}

\usepackage{mineira}

%- if problem.blocks.preamble is defined
	\VAR{problem.blocks.preamble}
%- endif

%- if vars.editorial is truthy
\AddToHook{shipout/foreground}{
       \begin{tikzpicture}[overlay, remember picture]
            \node[text=gray, rotate=45, scale=8, opacity=0.3]
				at (0.5\paperwidth,-0.6\paperheight) {EDITORIAL};
        \end{tikzpicture}
    }
%- endif

\begin{document}

\includeProblem
%- if problem.short_name is defined
	[\VAR{problem.short_name}]
%- endif
{\VAR{problem.title | escape}}{

	\VAR{problem.blocks.legend}

	%- if problem.blocks.input is defined
		\inputdesc{\VAR{problem.blocks.input}}
	%- endif

	%- if problem.blocks.output is defined
		\outputdesc{\VAR{problem.blocks.output}}
	%- endif

	%- if problem.blocks.interaction is defined
		\interactiondesc{\VAR{problem.blocks.interaction}}
	%- endif

	%- if problem.samples
		\vspace{0.3cm}
		\setboolean{widesamples}{false}

		% Find out if should be wide-samples or not
		%- for sample in problem.samples
			%- if sample.interaction is none
				\checkIfWide{\VAR{sample.inputPath}}
				\checkIfWide{\VAR{sample.outputPath if sample.outputPath is not none else ''}}
			%- endif
		%- endfor

		\subsection*{Exemplos}
		%- for sample in problem.samples
			%- if sample.interaction is none
				\exemploMaybeWide{\VAR{sample.inputPath}}
					{\VAR{sample.outputPath if sample.outputPath is not none else ''}}
			%- else
				\exemploInterativo
				%- for entry in sample.interaction.entries
					\interaction{\VAR{entry.data}}{\VAR{entry.pipe}}
				%- endfor
			%- endif

			%- if sample.explanation is not none
				\explanation{\VAR{sample.explanation}}
			%- endif
		%- endfor
	%- endif

	%- if problem.blocks.notes is defined
		\subsection*{Observações}
		\VAR{problem.blocks.notes}
	%- endif
}
%- if problem.blocks.editorial is defined
\subsection*{Solução}
\VAR{problem.blocks.editorial}
%- endif
\end{document}

