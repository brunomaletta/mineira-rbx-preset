\section*{Informações gerais}

Este caderno de tarefas é composto por \pageref*{lastpage} páginas (não
contando a capa), numeradas 
de 1 a \pageref*{lastpage}. Verifique se o caderno está completo.

\subsection*{Entrada}

\begin{itemize}
	\item A entrada deve ser lida da entrada padrão.

	\item A entrada consiste em exatamente um caso de teste, que é descrito usando uma
	quantidade de linhas que depende do problema. O formato da entrada é como descrito
	em cada problema. A entrada não contém nenhum conteúdo extra.

	\item Todas as linhas da entrada, incluindo a última, terminam com o caractere de fim
	de linha (\texttt{\textbackslash n}).

	\item A entrada não contém linhas vazias.

	\item Quando a entrada contém múltiplos valores separados por espaços, 
	existe exatamente um espaço em branco entre dois valores consecutivos na mesma linha.
\end{itemize}

\subsection*{Saída}

\begin{itemize}
	\item A saída deve ser escrita na saída padrão.

	\item A saída deve respeitar o formato especificado no enunciado. A saída não deve
	conter nenhum dado extra.

	\item Todas as linhas da saída, incluindo a última, devem terminar com o caractere de fim
	de linha (\texttt{\textbackslash n}).

	\item Quando uma linha da saída apresentar múltiplos valores separados por espaços, deve haver
	exatamente um espaço em branco entre dois valores consecutivos.

	\item Quando um valor da saída for um número real, use pelo menos o número de casas decimais correspondente
	à precisão requisitada no enunciado.
\end{itemize}

\subsection*{Problemas Interativos}

A prova pode conter problemas interativos. Nesse tipo de problema, os dados de entrada fornecidos ao seu programa podem não ser predeterminados, mas são construídos especificamente para a sua solução. O juiz escreve um programa especial (o interador), cuja saída é transferida para a entrada da sua solução, e a saída do seu programa é enviada para a entrada do interador. Em outras palavras, sua solução e o interador trocam dados e podem decidir o que imprimir com base no histórico da comunicação.

Quando você escreve a solução para um problema interativo, é importante lembrar que, se você imprimir algum dado, é possível que ele seja primeiro armazenado em um \textit{buffer} interno e não seja transferido imediatamente para o interador. Para evitar essa situação, você deve usar uma operação especial de \textit{flush} toda vez que imprimir algum dado. Essas operações de \textit{flush} estão presentes nas bibliotecas padrão de quase todas as linguagens:

\begin{itemize}
	\item \texttt{fflush(stdout)} em \texttt{C}.
	\item \texttt{cout.flush()} em \texttt{C++}.
	\item \texttt{sys.stdout.flush()} em \texttt{Python}.
	\item \texttt{System.out.flush()} em \texttt{Java}.
\end{itemize}
