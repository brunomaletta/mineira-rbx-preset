\documentclass{article}
\usepackage[a4paper, margin=1in]{geometry}
\usepackage{multirow}
\usepackage{multicol}
\usepackage{graphicx}
\usepackage{amsfonts}
\usepackage[brazil]{babel}
\usepackage[utf8]{inputenc}
\usepackage[T1]{fontenc}

\usepackage[utf8]{inputenc}

\pagestyle{myheadings}
\markright{\VAR{contest.title}}

\begin{document}

\begin{titlepage}
\begin{center}

\includegraphics[width=4cm]{logo.png}

\vspace{1cm}
{\huge{\bf \VAR{vars.year}}} \\

\vspace{0.5cm}
{\huge \bf Folha de Informações}\\[12pt]
{\Large \bf \VAR{contest.title}}
\end{center}

\section{Informações sobre a duração da prova}

\begin{itemize}
	\item A prova tem duração de 5 horas.
	\item Uma hora antes do fim da prova, o placar será congelado e não será mais atualizado.
	\item Durante os últimos 15 minutos de prova, você não receberá o veredito de suas submissões, mas elas ainda serão contabilizadas para o placar final.
\end{itemize}

\section{Informações sobre o Ambiente de Teste}

\subsection{Ambiente}

O sistema de correção de submissões será executado na distribuição \texttt{Ubuntu GNU/Linux 22.04 LTS amd64}, com as seguintes configurações.

\subsection{Linguagens aceitas}

\begin{multicols}{2}
\begin{itemize}
    \item \texttt{C}: gcc versão 11.4.0
    \item \texttt{C++20}: gcc versão 11.4.0
    \item \texttt{Python}: Python 3.10.12
    \item \texttt{Java}: openjdk 17.0.11
\end{itemize}
\end{multicols}

\subsection{Limites de tempo e memória}

\begin{center}
\begin{tabular}{c|ccc|c}
& \multicolumn{3}{c|}{Tempo (s)} & \multirow{2}{*}{Memória (MiB)} \\
%\cline{2-4}
{\sf Problema} & {\sf C/C++} &{\sf Java} & {\sf Python3} & \\
\hline
%- for problem in problems:
\VAR{problem.short_name}
& \VAR{problem.limits.timelimit_for_language('cpp') / 1000 | round(1, 'floor')}
& \VAR{problem.limits.timelimit_for_language('java') / 1000 | round(1, 'floor')}
& \VAR{problem.limits.timelimit_for_language('py') / 1000 | round(1, 'floor')}
& \VAR{problem.limits.memoryLimit}
\\ \hline
%- endfor
\end{tabular}
\end{center}

\noindent Antes da competição, os juízes terão resolvido todos os problemas na linguagem \texttt{C}/\texttt{C++}. Os limites de tempo para cada problema serão calculados com base no tempo de execução dessas soluções.

\subsection{Outros Limites}

\begin{itemize}
    \item Tamanho do arquivo fonte: \texttt{100KiB}
\end{itemize}

\subsection{Comandos de Compilação}

\begin{itemize}
    \item \texttt{C}: \texttt{gcc -x c -O2 -std=gnu11 -static -lm}
    \item \texttt{C++20}: \texttt{g++ -x c++ -O2 -std=gnu++20 -static}
    \item \texttt{Java}: \texttt{javac}
\end{itemize}

\subsection{C/C++20}

\begin{itemize}
    \item Seu programa deve retornar zero, executando, como último comando, \texttt{return 0} ou \texttt{exit(0)}.
\end{itemize}

\subsection{Java}

\begin{itemize}
    \item Não declare um \texttt{package} no seu programa Java; caso contrário, o programa não será executado no Boca.
    \item Comando para execução:

	\texttt{java -Xms\{limite\_memoria\}m -Xmx\{limite\_memoria\}m -Xss256m \{codigo\_problema\}}.
    \item Atenção: soluções em Java podem não cumprir os limites de tempo em alguns problemas devido ao desempenho da linguagem.
\end{itemize}

\subsection{Python}

\begin{itemize}
    \item Apenas \texttt{Python 3} é suportado.
	\item Programs em Python são interpretados. Erros de sintaxe resultarão em ``\texttt{Runtime Error}''.
    \item Atenção: soluções em Python podem não cumprir os limites de tempo em alguns problemas devido ao desempenho da linguagem.
\end{itemize}

\section{Instruções para o Uso do Sistema de Submissão Boca}

\subsection{Submissão de Soluções}

Para submeter uma solução, use a interface web do Boca:
\begin{enumerate}
    \item Abra seu navegador.
    \item Faça login como equipe (usuário e senha fornecidos).
    \item Acesse a aba \texttt{Runs}. Selecione o problema, a linguagem e envie o arquivo.
\end{enumerate}

\noindent Os vereditos que você pode receber do juiz são:

\begin{itemize}
    \item \texttt{YES}: sua submissão foi aceita.
    \item \texttt{NO - Compilation error}: sua submissão teve um erro de compilação.
    \item \texttt{NO - Runtime error}: sua submissão teve erro em tempo de execução
    \item \texttt{NO - Time limit exceeded}: sua submissão demorou mais que o tempo limite do problema para executar.
    \item \texttt{NO - Wrong answer}: sua submissão deu resposta errada.
    \item \texttt{NO - Contact staff}: ocorreu um erro inesperado. Use a aba \texttt{Clarifications}.
\end{itemize}

\noindent Se houver múltiplos erros (como ``\texttt{Time limit exceeded}'' e ``\texttt{Wrong answer}''), qualquer um pode ser retornado.

\subsubsection{Observações}

\begin{itemize}
    \item Não há vereditos como ``\texttt{Presentation Error}''. Erros de formatação (como escrever ``impossivel'' em vez de ``impossível'') resultam em ``\texttt{Wrong Answer}''.
    \item Espaços em branco extras (dentro do razoável) são aceitos, mas exageros (como milhares de espaços) resultam em ``\texttt{Wrong Answer}''.
    \item Saídas que excederem o limite de tamanho especificado resultarão em ``\texttt{Runtime Error}'' (isso inclui o conteúdo escrito em \texttt{stderr}).
\end{itemize}

\subsubsection{Penalidades}

\begin{itemize}
    \item Cada submissão com veredito ``\texttt{NO}'' antes do primeiro ``\texttt{YES}'' adiciona \textbf{20 minutos} ao tempo total da equipe.
    \item Exceções para essa regra são os vereditos ``\texttt{NO - Compilation error}'' e ``\texttt{NO - Contact staff}'', que não geram penalidade.
\end{itemize}

\subsubsection{Tempo de Resposta}

\begin{itemize}
    \item O tempo para obter um veredito varia conforme o problema e o momento da competição.
    \item Submissões podem ser julgadas fora de ordem, e verificações manuais podem causar atrasos.
\end{itemize}

\subsection{Clarifications}

Toda comunicação com os juízes é feita via mensagens de esclarecimento (\texttt{Clarifications}):

\begin{enumerate}
    \item Acesse a aba \texttt{Clarifications}.
    \item Selecione o problema e envie sua dúvida.
\end{enumerate}

\noindent Respostas dos juízes e perguntas enviadas aparecerão nessa aba. \textbf{Leia atentamente o enunciado antes de pedir esclarecimentos}. Os juízes não revelerão informações extras sobre os casos de teste ou o motivo de um veredito.

\subsection{Placar}

Para ver o placar:

\begin{enumerate}
    \item Acesse a aba \texttt{Score}.
    \item O placar local será exibido.
\end{enumerate}

\subsection{Tasks}

Na aba \texttt{Tasks}, a equipe pode:

\begin{itemize}
    \item Enviar arquivos para impressão (\texttt{Send}).
    \item Pedir ajuda técnica (botão ``\texttt{S.O.S.}''). A organização só auxilia com problemas técnicos (hardware, etc.).
\end{itemize}

\end{titlepage}
\end{document}
